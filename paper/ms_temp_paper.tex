% DOCUMENT PREAMBLE
\documentclass{emulateapj}
% \documentclass[referee]{apj}

\usepackage{graphicx}
\usepackage{float}

\usepackage{subcaption}

\usepackage{apjfonts}
\usepackage{microtype}

\begin{document}

\title{Solar jet observed at the limb in SDO/SST}
\author{S.M.Bennett$^1$
and
\author{A.J.Leonard$^1$}
\affil{$^1$ Solar Physics and Space Plasma Research Centre (SP2RC), University of Sheffield, Hicks Building, Hounsfield Rd, S3 7RH, UK}


\begin{abstract}
\end{abstract}

\maketitle

\section{Introduction}
Solar jets permeate the solar atmosphere, from spicules low in the chromosphere to macropsicules passing through the transition region and X-Ray jets extending into the solar corona. Investigations into these phenomena have advanced significantly with recent developments in solar telescopes such as Hinode and the Solar Dynamic Observatory. 


\section{Observation}
We observed a jet like feature at the limb on $21st$ June 2016 beginning at $07:30:00$ in CRISP, an instrument installed on the Swedish Solar Telescope (SST) during a period of good seeing. %cite scharmer et al 2003
We used the H$\alpha$ filter, core line $656.28$ nm with $35$ slit increments from the core covering a $.32$ nm range, $-0.2$ and $+0.12$, and were further processed using te Multi-Object Multi-Frame Blind Deconvolution (MOMBFD cite). % van Noort et al. 2005
The observations were of active region AR 11506 with $xc = 893, yc = -250$ in heliocentric coordinates on $930x930$ pixel images, with spatial resolution on $0.012$ arcsec/pixel and temporal resolution on $7.5$ sec.
Due to the constant surveillance we currently have the Sun under, we also have simultaneous observations with the Solar Dynamic Observatory (SDO) and the Solar Terrestrial Relations Observatory (STEREO).
Using the Atmospheric Imaging Assembly (AIA), we observe the jet in most of the wavelengths available, $30.4$, $35.5$, $211$, $17.1$ and $13.1$ nm.
AIA on-board SDO provides $4096 \times 4096$ pixel images with a spatial resolution of $0.6$ arcsec per pixel and a cadence of $12$ sec.
Lastly, we also have observations in STEREO using HIA. 
We are fortunate that when these observations were taken, STEREO A was at approximately $90^\circ$ to the Sun-Earth line, as such we also have observations of this feature as an on-disk feature. % need a citiation for STEREO
In this case we are using the $30.4$ nm HIA instrument, however, the distance from the Earth has now reached a point that the temporal cadence has reduced to $10$ min.
While this is possibly too high to undertake a detailed examination, we can certainly utilise this method to inform us as to the global behaviour of the macrospicule.   

\section{Results}

\section{Discussion}

\section{Conclusion}



\bibliographystyle{apj}
\bibliography{references}







\end{document}